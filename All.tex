\documentclass[a4paper,11pt]{article}
\usepackage[margin=1in, left=2.5in]{geometry}
\RequirePackage[colorlinks,citecolor=blue,urlcolor=blue]{hyperref}
\usepackage[T1]{fontenc}
\usepackage[utf8]{inputenc}
\usepackage{amsmath,amssymb,amsthm,mathrsfs}
\usepackage[utf8]{inputenc}
\usepackage{babel}
\usepackage{url}
\usepackage[strict=true,style=english]{csquotes}
% \usepackage[backend=biber, style=authoryear]{biblatex}
% \usepackage[citestyle=alphabetic,bibstyle=authortitle]{biblatex}
\usepackage[style=authoryear,
            bibstyle=authoryear,
            citestyle=authoryear,
            natbib=true,
            hyperref=true,
            backref=true,
            abbreviate=true]{biblatex}
\usepackage{citeall}
\usepackage[notref,notcite]{showkeys}

\addbibresource{All.bib}

\def\polhk#1{\setbox0=\hbox{#1}{\ooalign{\hidewidth
  \lower1.5ex\hbox{`}\hidewidth\crcr\unhbox0}}} \def\cprime{$'$}

\title{My feference bank}
\author{Le Chen\footnote{Email: \url{le.chen@auburn.edu}, \url{chenle02@gmail.com}.}}

\date{\today}

\begin{document}

\tableofcontents

\section{Introduction}

\subsection{Source}
Here is a reference bank. The biblatex entries were mostly downloaded from

\begin{center}
  \url{https://mathscinet.ams.org/mathscinet}
\end{center}

\noindent and from the \textit{arXiv}.

\subsection{Naming convention}

The naming convention consists of three cases:
\begin{enumerate}
  \item Single authored paper, such as:
  \item[] Einstein, Albert. Random PDE for special relativities.
    \textit{Annals of Probability}, Volume, Number, 2023.
    \begin{center}
      einstein:23:random
    \end{center}
    \bigskip

  \item Paper with two authors, such as:
  \item[] Einstein, Albert and Grothendieck, Alexandre. A stochastic PDE
    model for general relativities. \textit{Electronic Journal of Probability},
    Volume, Number, 2024.
    \begin{center}
      einstein.grothendieck:24:stochastic
    \end{center}
    
  \item Paper with more than two authors, such as:
  \item[] Einstein, Albert and Grothendieck, Alexandre and Newton, Isaac.
    A private communication on interemittency. \textit{Transactions of AMS},
    Volume, Number, 2025.
    \begin{center}
      einstein.grothendieck.ea:25:private
    \end{center}
\end{enumerate}

Here is a demonstration how to use it in neovim:
\begin{center}
  \url{https://asciinema.org/a/596819}
\end{center}

\begin{center}
  \url{https://mathscinet.ams.org/mathscinet}.
\end{center}

% \cite{MR428626}

% \printbibliography[resetnumbers=0]

% \printbibliography

% \printbibheading
% \printbibliography[keyword=major,heading=subbibliography,title={Major Sources}]
% \printbibliography[keyword=minor,heading=subbibliography,title={Minor Sources}]
\tableofcontents

\citeall

\section{Articles}%
\label{sec:Articles}

\printbibliography[type=article,title={Articles}]
\section{Books}%
\label{sec:Books}

\printbibliography[type=book,title={Books}]

\section{In proceedings}%
\label{sec:In proceedings}

\printbibliography[type=inproceedings,title={In proceedings}]

\section{In collections}%
\label{sec:In collections}

\printbibliography[type=incollection,title={In collection}]

% \printbibliography[keyword={physics},title={Physics-related only}]
% \printbibliography[keyword={latex},title={\LaTeX-related only}]

\end{document}
